\chapter{Conclusões}\label{cap5_conclusoes}

{ Nos capítulos anteriores foram explorados conceitos sobre organização e arquitetura
    de computadores, \textit{overviews} de arquiteturas famosas e um aprofundamento
    na especificação do conjunto de instruções \textit{RISC-V}. Além disso, foi
    descrito o que é, como funciona e quais são as peças do Ambiente de Aprendizado
    de microarquiteturas Uniciclo, Multiciclo e Pipeline, o \textit{RISC-V SiMPLE},
    desde o uso de ferramentas preexistentes até a criação de materiais próprios.
}


\section{Objetivos Alcançados}
    { O projeto de melhoria e documentação da plataforma utilizada no laboratório
        obteve resultados satisfatórios. Tanto a presença da documentação da
        arquitetura quanto as melhorias em código devem permitir que o uso da
        ferramenta seja facilitado.
    }

    { A implementação do conjunto de instruções \textit{RV32IMF} em \textit{pipeline}
        de cinco estágios ainda apresenta alguns \textit{bugs} de execução.
        O trabalho conseguiu corrigir alguns desses erros, mas a implementação
        ainda não se encontra no estado de ter a garantia de funcionamento correto
        do código ao executar operações de ponto flutuante sem a inserção de
        \texttt{nop}s no código.
    }

    { Em relação às oito outras implementações, não foram encontrados erros nos
        testes realizados, e seu uso pode ser considerado seguro.
    }

    { A expansão das ferramentas de auxílio ao desenvolvimento e depuração como
        a automação do processo de síntese de todas as versões do processador e
        o uso de ferramentas mais completas como o \textit{GTKWave} trouxeram
        maior observabilidade dos problemas, suas causas e caminhos para solucioná-los,
        acelerando o ciclo de desenvolvimento do projeto. Com isso, o resultado
        final se alinha à maioria das suas expectativas.
    }

\clearpage
\section{Perspectivas Futuras}
    { Apesar do trabalho ter tido bom progresso, ainda existem diversas possíveis
        melhorias, como:
    }
    \begin{itemize}
        \item Deixar a implementação do \textit{pipeline bug-free};
        \item Simplificar partes do projeto de \textit{hardware} para melhorar o desempenho do sistema;
        \item Implementar uma versão de 64 \textit{bits} do sistema;
        \item Implementar uma \textit{ISA} diferente usando a plataforma como base, permitindo a escolha da arquitetura pelo arquivo \texttt{config.v}.
    \end{itemize}


\section{Palavras Finais}

    { Com o recente sucesso dos processadores \textit{ARM M1} lançados pela
        \textit{Apple}, e com os processadores \textit{ARM Graviton} disponíveis
        no serviço de servidores em nuvem da \textit{Amazon}, é uma possibilidade
        forte que o desenvolvimento de plataformas \textit{RISC-V} para uso geral
        desacelere e o mercado \textit{ARM} cresça ainda mais. De um ponto de vista
        mercadológico, o desenvolvimento de uma plataforma de aprendizagem em
        \textit{ARM} aumentaria as chances de aplicação direta do conteúdo aprendido
        no ambiente de trabalho.
    }

    { Atualmente, algumas versões de seu conjunto de instruções foram abertas.
        Porém, a incerteza quanto a possíveis problemas de licenciamento para
        desenvolvimento e distribuição de uma solução \textit{ARM} no início do
        projeto inviabilizaram seu uso para o presente trabalho.
    }

    { Mesmo assim, a \textit{RISC-V} já possui um bom alicerce, tem futuro promissor
        e a escolha da \textit{ISA} para o desenvolvimento do trabalho se mostrou
        uma decisão acertada. Mercados emergentes também abrem diversas portas e
        se espera que o conhecimento adquirido desenvolvendo e utilizando a
        plataforma se provará de grande valia.
    }

