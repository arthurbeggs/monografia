\chapter{Introdução}\label{CapIntro}

%\resumodocapitulo{Resumo opcional}

\section{Motivação}
{
    O mercado de trabalho está a cada dia mais exigente, sempre buscando
    profissionais que conheçam as melhores e mais recentes ferramentas
    disponíveis. Além disso, muitos universitários se sentem desestimulados
    ao estudarem assuntos desatualizados e com baixa possibilidade de
    aproveitamento do conteúdo no mercado de trabalho. Isso alimenta o
    desinteresse pelos temas abordados e, em muitos casos, leva à evasão
    escolar. Assim, é importante renovar as matérias com novas tecnologias
    e tendências de mercado sempre que possível, a fim de instigar o
    interesse dos discentes e formar profissionais mais capacitados e
    preparados para as demandas da atualidade.
}

{
    Hoje, a disciplina de Organização e Arquitetura de Computadores da
    Universidade de Brasília é ministrada utilizando a arquitetura
    \textit{MIPS32}. Apesar da arquitetura \textit{MIPS32} ainda ter
    grande força no meio acadêmico (em boa parte devido a sua simplicidade
    e extensa bibliografia), sua aplicação na indústria tem diminuído
    consideravelmente na última década.
}

{
    Embora a curva de aprendizagem de linguagens \textit{Assembly} de
    alguns processadores \textit{RISC} seja relativamente baixa para quem
    já  conhece o \textit{Assembly MIPS32}, aprender uma arquitetura atual
    traz o benefício de conhecer o \textit{estado da arte} da organização e
    arquitetura de computadores.
}

{
    Para a proposta de modernização da disciplina, foi escolhida a 
    \textit{ISA RISC-V} desenvolvida na Divisão de Ciência da Computação da
    Universidade da Califórnia, Berkeley como substituta à
    \textit{ISA MIPS32}.
}


\section{Por que \textit{RISC-V}?}
{
    A \textit{ISA RISC-V} (lê-se \textit{``risk-five''}) é uma arquitetura
    \textit{open source} com licença \textit{BSD}, o que permite o seu
    livre uso para quaisquer fins, sem distinção de se o trabalho possui
    código-fonte aberto ou proprietário. Tal característica possibilita que
    grandes fabricantes utilizem a arquitetura para criar seus produtos,
    mantendo a proteção de propriedade intelectual sobre seus métodos de
    implementação e quaisquer subconjuntos de instruções
    não-\textit{standard} que as empresas venham a produzir, o que
    estimula investimentos em pesquisa e desenvolvimento.
}

{
    Empresas como Google, IBM, AMD, Nvidia, Hewlett Packard, Microsoft,
    Qualcomm e Western Digital são algumas das fundadoras e investidoras
    da \textit{RISC-V Foundation}, órgão responsável pela governança da
    arquitetura. Isso demonstra o interesse das gigantes do mercado no
    sucesso e disseminação da arquitetura.
}

{
    A licença também permite que qualquer indivíduo produza, distribua e
    até mesmo comercialize sua própria implementação da arquitetura sem ter
    que arcar com \textit{royalties}, sendo ideal para pesquisas
    acadêmicas, \textit{startups} e até mesmo \textit{hobbyistas}.
}

{
    O conjunto de instruções foi desenvolvido tendo em mente seu uso em
    diversas escalas: sistemas embarcados, \textit{smartphones},
    computadores pessoais, servidores e supercomputadores, o que permitirá
    maior reuso de \textit{software} e maior integração de
    \textit{hardware}.
}

{
    Outro fator que estimula o uso do \textit{RISC-V} é a modernização dos
    livros didáticos. A nova versão do livro utilizado em OAC, Organização
    e Projeto de Computadores, de David Patterson e John Hennessy, utiliza
    a \textit{ISA RISC-V}.
}

{
    Além disso, com a promessa de se tornar uma das arquiteturas mais
    utilizadas nos próximos anos, utilizar o \textit{RISC-V} como
    arquitetura da disciplina de OAC se mostra a escolha ideal no momento.
}

\section{O Projeto \textit{RISC-V SiMPLE}}
{
    O projeto \textit{RISC-V SiMPLE (Single-cycle Multicycle Pipeline
    Learning Environment)} consiste no desenvolvimento de um processador
    com conjunto de instruções \textit{RISC-V}, sintetizável em
    \textit{FPGA} e com \textit{hardware} descrito em \textit{Verilog}. A
    microarquitetura implementada nesse trabalho é uniciclo, escalar, em
    ordem, com um único \textit{hart} e com caminho de dados de 64 bits.
    Trabalhos futuros poderão utilizar a estrutura altamente configurável
    e modularizada do projeto para desenvolver as versões em
    microarquiteturas multiciclo e \textit{pipeline}.
}

{
    O processador contém o conjunto de instruções I (para operações com
    inteiros, sendo o único módulo com implementação mandatória pela
    arquitetura) e as extensões \textit{standard} M (para multiplicação e
    divisão de inteiros) e F (para ponto flutuante com precisão simples
    conforme o padrão IEEE 754 com revisão de 2008). O projeto não
    implementa as extensões D (ponto flutuante de precisão dupla) e A
    (operações atômicas de sincronização), e com isso o \textit{soft core}
    desenvolvido não pode ser definido como de propósito geral, G (que deve
    conter os módulos I, M, A, F e D). Assim, pela nomenclatura da
    arquitetura, o processador desenvolvido é um \textit{RV64IMF}.
}

{
    O projeto contempla \textit{traps}, interrupções, exceções,
    \textit{CSRs}, chamadas de sistema e outras funcionalidades de nível
    privilegiado da arquitetura.
}

{
    O \textit{soft core} possui barramento Avalon para se comunicar com os
    periféricos das plataformas de desenvolvimento. O projeto foi
    desenvolvido utilizando a placa DE2--115 com \textit{FPGA Altera
    Cyclone} e permite a fácil adaptação para outras placas da Altera.
}

